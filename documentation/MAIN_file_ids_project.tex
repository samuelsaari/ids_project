
%%--------------------------------------------------

%READ ME 
% Use this like Google docs. Track changes should be on (for you). Make any changes you desire (or comment) and I will make sure it compiles nicely should you encounter any issues.

% This short video will help you get started if needed:
% https://youtu.be/S6Si-F5ArIw 
% The video is for an old article but the principle is the same

% Cite like this: (Smith 2021 "Article title") and I will look it up.

% Note that the preamble is in the 0dPreamble_..._Maki.sty file
% ---> Preamble includes settings, author information, affiliations, and title of the document.

%%%-----------------------------------------------


%%%-----------------------------------------------
% Some settings that cannot be included in the 0Preamble-file

\RequirePackage{snapshot} % creates a .dep file of the dependencies
\documentclass[a4paper,12pt,bibliography=totoc,numbers=noenddot,sfdefaults=false,abstract=true,notitlepage]{scrartcl} % NB!  % NB! TexPublish reguires this to be on a single line
\usepackage{0Preamble_ids}

% include all figures inserted with \InsertFloat in texcount
%TC:macroword \InsertFloat [float]

% do not count stuff inside begin{comment/B}..end{comment/B} 
%TC:group comment 0 0
%TC:group B 0 0

%%%%-----------------------------
\addbibresource{0MyLibrary.bib}

\begin{document} % Document starts here
	
	
	\begin{singlespace}
	\maketitle % Brings front page information from the 0Preamble-file
		
	% 	\begin{abstract}
	% 		{\parindent0pt % disables indentation for all the text between { and }
				
	% 			\blindtext
				
	% 			\keywords{xx $\cdot$ yyy  }
	% 		}% Indent end
	% 		consider summarising the main research questions /. hypotheses in the abstract -- at least 90% of people read only the abstract.
	% 	\end{abstract} \hspace{10pt}
		
	\end{singlespace}


	% \newpage
	
	%%%%-----------------------------------------------
	%%%% Introduction
	
	%\clearpage %OBS!!
	\section{Introduction}\label{intro}
	
	%%%% Define a research territory (1)
	%% General context of the work (1a)  
	%% Narrower research area and statement of its importance (1b)
	Play\textemdash coming together to engage in a common recreational activity\textemdash is a human universal \autocite{brownHumanUniversalsHuman2004} and essential part in human development \autocite{smithPlayTypesFunctions2005,pellegriniRolePlayHuman2009}. Board games are a popular and accessible form of play that bring family, friends and strangers alike together, and promote well-being across the life-span \autocite{dellangelaBoardGamesEmotional2020,solway2011wellness}.

	%%%% Establish a niche (2)
	%% Identification of a gap or other need for research (2a)
	% Specific research question meeting the identified need (2b)
	But what if a group of people have outplayed the games they own and would like to find some new ones? There are at least 150.000 different board games out there \autocite{wordsratedBoardGamesStatistics2025} so finding a good match can be a time consuming process. A service that would suggest new board games based on the individual taste could come in handy on such occasions.
	
	%%% Summary
	% Summary of approach to answer the research question 3a
	%Announcement of principal findings
	This technical report will outline the steps we made to create a recommendation system for new board games based on the user's ratings in BoardGameGeek.com (BGG) database as well as the technical details of setting up a webpage *LINK TO WEBSITE* through which users can find recommendations for new board games. To achieve this, we used non-negative matrix factorization (NMF) with a selected users' review scores to create tailored suggestions for new board games.
	
	
	%%%%-----------------------------------------------
	%%%% Data and Methods
	
	
	\section{Data}\label{data}

	%BGG Data
	We chose the biggest board game ratings database BoardGameGeek.com to fetch user ratings and board game metadata, such as playing mechanics and category. As it is openly available, it having a data access was simple and there were no data privacy issues to be addressed.

	Since there are 2.7 million users  and 150.000 boardgames in BGG \autocite{didymus-trueBoardGameGeeksSupportDrive2024,wordsratedBoardGamesStatistics2025}, fetching all of them would not be feasible with the API interface, at least within this project. Since many users have left no or few reviews, the data would be too sparse if we had taken a random sample. To get a compromise between training matrix sparsity, time constraints and selection bias, we decided to choose x guilds (online discussion groups) that represent different geographic regions, target audiences (teens, parents, seniors), genres as well as general and special interest groups. This way we could get X reviews from Y reviewers about Z games. The sampling strategy did bias our distribution towards active users, but also allows for reliable training. Users with only few reviews would make the estimation computationally intensive and possibly unreliable.
	
	We decided to load our data via the BGG XML API \autocite{bbgBGGXMLAPI22025}. While the data is well preprocessed and clean, getting the data was no simple task.

	First, the API has rate limits that we had to find out by trial and error. We used a base delay of .75 seconds and if the rate was limited, increased the delay exponentially and waited for the maximum of one minute. The maximum batch size for games was 20 and it took some time to find out why the game metadata coverage was so low on larger test runs.

	Second, since the game metadata fetching was unreliable even with the exponential backoff time, we used a game metadata cachefile to make sure that if a game metadata had been fetched successfully once, it was stored  on the project home directory.

	Finally, we included only users with at least 25 reviews and fetched the most popular games first to avoid training data matrix sparsity.
	
	
	\section{Methods}\label{methods}

	Using review scores is the single best way of generating recommendations \autocite{epsteinRangeWhyGeneralists2021}. It seems counterintutive as one would think that categories, genres, sales or other available metadata would be equally helpful. Based on this empirical finding and the scope of the project, we chose to focus solely on the review scores.

	
	
	%%%%-----------------------------------------------
	%%%% Results
	
	\section{Results}\label{results}
	
	
	
	
	
	
	
	
	
	%%%%-----------------------------------------------
	%%%% Discussion and conclusion
	
	\section{Discussion}\label{discussion}

	We created a board game recommendation webpage based on a sample of 200.000 board game reviews from 1000 active BoarGameGeek.com users to help BoardGameGeek users\textemdash or anyone\textemdash find new boardgames that match their taste.
	
	%%%% Interpretation of results to answer research question
	
	%% Surprising results?
	
	%%%% Possible weaknesses
	
	%%%% Broader implications / Comparison or synthesis with results from literature
	%% how does relate to other research questions?
	%% does it support current hypotheses in your field?
	%% how does it relate with literature (wider than topic of this paper)
	
	
	%%%% Prospects for future research
	If this project was expanded, we would use the whole BGG database as a training data %, and have an option to rate a curated list to get recommendations even without a BGG user account.
	
	% Conclusion for Article
	
	%\clearpage
	%%%--------cd C:\Users\mmak\OneDrive - Väestöliitto ry\FLH-THESIS\1_determinants\1publish_determinantscd C:\Users\mmak\OneDrive - Väestöliitto ry\FLH-THESIS\1_determinants\1publish_determinants---------------------------------------
	% Bibliography
	\FloatBarrier
	%\begin{spacing}{1.3} 
	\printbibliography
	%\end{spacing}
	
	
	%TC:ignore 
	% check that texcount works as intended
	
	%%%-----------------------------------------------
	%% Figures for Submit (S) version
	
	% Figures and tables that appear in the body here again for the submit (S) version

	
	\savepage{lastpage} % page count 
	
	%-------------------------------------------------------------
	% Appendix
	%-------------------------------------------------------------
	% First some settings, bear with me!
	\clearpage
	\setcounter{secnumdepth}{3} % add numbers to section headings, needed for prefix in the figures
	\appendix % start appendix
	%\counterwithin{figure}{section} % A prefix for figures
	%\counterwithin{table}{section} % A prefix for tables
	%\stepcounter{section} % stars numbers from 1 onwards (A.1 etc)
	
	% Custom prefixes
	\renewcommand{\thefigure}{S\arabic{figure}}
	\setcounter{figure}{0}
	\renewcommand{\thetable}{S\arabic{table}}
	\setcounter{table}{0}
	
	\pagenumbering{arabic} % page count in arabic numbers
	\renewcommand{\thepage}{Appendix \Roman{page}} % A. prefix for  page numbering
	
	\addsec{Appendix}\label{------APPENDIX------} % KOMA-class section without numbers (same as \section*{}), but will keep the "A."-prefixes in floats and include the section in table of contents
	
	\FloatBarrier
	
	\clearpage
	
	
	
	
	%TC:endignore
	\savepage{applastpage} % page count
	
\end{document}


%%%%%%%%%%%%%%%%%%%%%%%%%%%%%%%%%%%%%%%%%%%%%%%%%%%%%%%%%%%%%%%%%%%%
%%---------------------
